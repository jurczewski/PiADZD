\documentclass{classrep}
\usepackage[utf8]{inputenc}
\frenchspacing

\usepackage{graphicx}
\usepackage[usenames,dvipsnames]{color}
\usepackage[hidelinks]{hyperref}
\usepackage{float}
\usepackage[table,xcdraw]{xcolor}
\usepackage{multirow}
\usepackage[table,xcdraw]{xcolor}

\usepackage{amsmath, amssymb, mathtools}

\usepackage{fancyhdr, lastpage}
\pagestyle{fancyplain}
\fancyhf{}
\renewcommand{\headrulewidth}{0pt}
\cfoot{\thepage\ / \pageref*{LastPage}}

\renewcommand{\refname}{Bibliografia}

% bullet itemize
\renewcommand{\labelitemi}{\textbullet}

\studycycle{Informatyka stosowana, studia dzienne, II st.}
\coursesemester{II}

\coursename{Przetwarzanie i analiza dużych zbiorów danych}
\courseyear{2020/21}

\courseteacher{mgr inż. Rafał Woźniak}
\coursegroup{środa, 11:45}

\author{%
\\
  \studentinfo[234128@edu.p.lodz.pl]{Piotr Wardęcki}{234128}\\
  \studentinfo[234053@edu.p.lodz.pl]{Paweł Galewicz}{234053}\\
  \studentinfo[234067@edu.p.lodz.pl]{Bartosz Jurczewski}{234067}%
}

\title{Zadanie 4}

\begin{document}
\maketitle
\thispagestyle{fancyplain}
\clearpage

\section{Cel zadania}
Celem zadania było napisanie programu, który implementuje reguły asocjacyjne za pomocą algorytmu A-priori. Program na podstawie ostatnio przeglądanych przez użytkowników przedmiotów identyfikuje te, które regularnie występowały wspólnie w ramach pojedynczej sesji.



%%%%%%%%%%%%%%%%%%%%%%%%%%%%%%%%%%%%%%%%%%%%  OPIS IMPLEMENTACJI
\section{Opis implementacji}
Do wykonywania zadania niezbędna była instancji \textit{Apache Spark}. Aby ograniczyć liczbę zainstalowanych środowisk skorzystaliśmy z odpowiedniego obrazu dla Dockera \cite{docker}, który zawierał także \textit{Jupyter Notebook}, \textit{Python} oraz \textit{Miniconda}. Dodatkowo aby ułatwić tworzenie środowiska do kolejnych zadań i między naszymi komputerami skorzystaliśmy z narzędzia \textit{Docker Compose} (nasz plik \cite{docker-compose}). Do stworzenia programu bardzo pomocna była funkcja parallelize() która odpowiedzialna była za stworzenie rozproszonego zbioru danych na podstawie listy kolekcji. Pozwalało to na wykorzystanie paradygmatu map-reduce do rozwiązywania większej części zagadnień algorytmu.

%%%%%%%%%%%%%%%%%%%%%%%%%%%%%%%%%%%%%%%%%%%%  WYNIKI
\section{Wyniki}
\begin{table}[H]
\centering
\caption{5 reguł asocjacyjnych o największej ufności dla par}
\begin{tabular}{|c|l|c|}
\hline
\rowcolor[HTML]{FFCE93} 
{\color[HTML]{000000} \textbf{Nr}} & \textbf{Element}        & \textbf{Ufność}    \\ \hline
1                                  & DAI93865 { [}FRO40251{]} & 1.0                \\ \hline
2                                  & GRO85051 {[}FRO40251{]} & 0.999176276771005  \\ \hline
3                                  & GRO38636 {[}FRO40251{]} & 0.9906542056074766 \\ \hline
4                                  & ELE12951 { [}FRO40251{]} & 0.9905660377358491 \\ \hline
5                                  & DAI88079 { [}FRO40251{]} & 0.9867256637168141 \\ \hline
\end{tabular}
\end{table}


\begin{table}[H]
\centering
\caption{5 reguł asocjacyjnych o największej ufności dla trójek}
\begin{tabular}{|c|l|c|}
\hline
\rowcolor[HTML]{FFCE93} 
{\color[HTML]{000000} \textbf{Nr}} & \multicolumn{1}{c|}{\cellcolor[HTML]{FFCE93}\textbf{Element}} & \textbf{Ufność} \\ \hline
1                                  & DAI23334 ELE92920 { [}DAI62779{]}                              & 1.0             \\ \hline
2                                  & DAI31081 GRO85051 {[}FRO40251{]}                              & 1.0             \\ \hline
3                                  & DAI55911 GRO85051 {[}FRO40251{]}                              & 1.0             \\ \hline
4                                  & DAI62779 DAI88079 { [}FRO40251{]}                              & 1.0             \\ \hline
5                                  & DAI75645 GRO85051 {[}FRO40251{]}                              & 1.0             \\ \hline
\end{tabular}
\end{table}

%%%%%%%%%%%%%%%%%%%%%%%%%%%%%%%%%%%%%%%%%%%% Wnioski
\section{Wnioski}

\begin{itemize}
    \item Stworzone przez nas reguły asocjacyjne mają najlepsza dokładność dla trojek.
    \item Najczęściej występującym następnikiem wygenerowanych reguł był [FRO40251].

\end{itemize}

\nocite{*}
\begin{thebibliography}{0}
    
    \bibitem{docker}
    \textsl{Jupyter Notebook Python, Spark Stack}
    \url{https://hub.docker.com/r/jupyter/pyspark-notebook}

    \bibitem{docker-compose}
    \textsl{Plik Docker Compose do zadania 2}
    \url{https://github.com/jurczewski/PiADZD/blob/master/zad2/docker-compose.yml}

    
\end{thebibliography}
\end{document}